% Options for packages loaded elsewhere
\PassOptionsToPackage{unicode}{hyperref}
\PassOptionsToPackage{hyphens}{url}
\documentclass[
]{article}
\usepackage{xcolor}
\usepackage[margin=1in]{geometry}
\usepackage{amsmath,amssymb}
\setcounter{secnumdepth}{-\maxdimen} % remove section numbering
\usepackage{iftex}
\ifPDFTeX
  \usepackage[T1]{fontenc}
  \usepackage[utf8]{inputenc}
  \usepackage{textcomp} % provide euro and other symbols
\else % if luatex or xetex
  \usepackage{unicode-math} % this also loads fontspec
  \defaultfontfeatures{Scale=MatchLowercase}
  \defaultfontfeatures[\rmfamily]{Ligatures=TeX,Scale=1}
\fi
\usepackage{lmodern}
\ifPDFTeX\else
  % xetex/luatex font selection
\fi
% Use upquote if available, for straight quotes in verbatim environments
\IfFileExists{upquote.sty}{\usepackage{upquote}}{}
\IfFileExists{microtype.sty}{% use microtype if available
  \usepackage[]{microtype}
  \UseMicrotypeSet[protrusion]{basicmath} % disable protrusion for tt fonts
}{}
\makeatletter
\@ifundefined{KOMAClassName}{% if non-KOMA class
  \IfFileExists{parskip.sty}{%
    \usepackage{parskip}
  }{% else
    \setlength{\parindent}{0pt}
    \setlength{\parskip}{6pt plus 2pt minus 1pt}}
}{% if KOMA class
  \KOMAoptions{parskip=half}}
\makeatother
\usepackage{longtable,booktabs,array}
\usepackage{calc} % for calculating minipage widths
% Correct order of tables after \paragraph or \subparagraph
\usepackage{etoolbox}
\makeatletter
\patchcmd\longtable{\par}{\if@noskipsec\mbox{}\fi\par}{}{}
\makeatother
% Allow footnotes in longtable head/foot
\IfFileExists{footnotehyper.sty}{\usepackage{footnotehyper}}{\usepackage{footnote}}
\makesavenoteenv{longtable}
\usepackage{graphicx}
\makeatletter
\newsavebox\pandoc@box
\newcommand*\pandocbounded[1]{% scales image to fit in text height/width
  \sbox\pandoc@box{#1}%
  \Gscale@div\@tempa{\textheight}{\dimexpr\ht\pandoc@box+\dp\pandoc@box\relax}%
  \Gscale@div\@tempb{\linewidth}{\wd\pandoc@box}%
  \ifdim\@tempb\p@<\@tempa\p@\let\@tempa\@tempb\fi% select the smaller of both
  \ifdim\@tempa\p@<\p@\scalebox{\@tempa}{\usebox\pandoc@box}%
  \else\usebox{\pandoc@box}%
  \fi%
}
% Set default figure placement to htbp
\def\fps@figure{htbp}
\makeatother
\setlength{\emergencystretch}{3em} % prevent overfull lines
\providecommand{\tightlist}{%
  \setlength{\itemsep}{0pt}\setlength{\parskip}{0pt}}
% Letra Arial
\usepackage{fontspec}
\setmainfont{Arial}

% Estilo con logo
\usepackage{fancyhdr}
\usepackage{graphicx}

\pagestyle{fancy}

% Limpia solo los encabezados
\fancyhead{} 

% Mantén el pie de página como quieras (ejemplo: número centrado)
\fancyfoot[C]{\thepage}

% Logo en el encabezado derecho
\fancyhead[R]{\includegraphics[width=0.15\textwidth]{logo.png}}
\usepackage{bookmark}
\IfFileExists{xurl.sty}{\usepackage{xurl}}{} % add URL line breaks if available
\urlstyle{same}
\hypersetup{
  hidelinks,
  pdfcreator={LaTeX via pandoc}}

\author{}
\date{\vspace{-2.5em}}

\begin{document}

\thispagestyle{empty} 
\begin{center}
\includegraphics[width=\textwidth,height=\textheight,keepaspectratio]{portada.png}
\end{center}

\newpage

\tableofcontents

\newpage

\subsection{Comité organizador}\label{comituxe9-organizador}

\begin{tabular}{ll}
\toprule
Responsables & Actividad académica\\
\midrule
Daniela Robles & Profesor investigador\\
Lucia G Morales Reyes & Profesor investigador\\
Federico Sánchez Quinto & Profesor investigador\\
Diego Ortega Delvecchyo & Secretario Académico\\
Iliana Martínez & Administración\\
\addlinespace
Jair Santiago García Sotelo & Técnico Académico\\
Alessandro López Hernández & Estudiante de Posgrado\\
Evelia Lorena Coss Navarrete & Postdoctorante\\
Valeria Vela & Licenciatura de Ciencias Genómicas\\
Emiliano Ferro & Licenciatura de Ciencias Genómicas\\
\addlinespace
Eglee Lomelín & Administración\\
Iván Eduardo Sedeño Jiménez & Estudiante de Posgrado\\
Maritrini Colón González & Postdoctorante\\
Carina Uribe & Técnico Académico\\
\bottomrule
\end{tabular}

\subsection{Objetivo de los días académicos}\label{objetivo-de-los-duxedas-acaduxe9micos}

La \textbf{tercera edición de los Días Académicos del LIIGH-UNAM}, que se llevará a cabo los días \textbf{5 y 6 de febrero de 2026}, tiene como propósito principal \textbf{difundir las investigaciones desarrolladas en el Laboratorio Internacional de Investigación sobre el Genoma Humano}.

Durante estas jornadas se presentarán \textbf{conferencias representativas de todos los grupos de investigación activos}, en las que se compartirán los avances más recientes en sus respectivas áreas. Entre las líneas de trabajo que actualmente se desarrollan en el LIIGH destacan: \textbf{genómica de poblaciones, genética del cáncer, paleogenómica, biología de sistemas, medicina de precisión, estadística, bioinformática, ecología y evolución}.

Además, el programa incluye \textbf{tres sesiones de pósters académicos}, elaborados por estudiantes de distintos niveles de formación y por posdoctorantes que realizan actividades de investigación en nuestra entidad, fortaleciendo así la participación y el intercambio académico en todos los niveles.

\includegraphics[width=4.16667in,height=\textheight,keepaspectratio]{foto_grupo.jpg}

\newpage

\subsection{Programa académico}\label{programa-acaduxe9mico}

Jueves 5 de febrero, 2026

\begin{table}[!h]
\centering
\begin{tabular}{lll}
\toprule
\textbf{Horario} & \textbf{Descripción} & \textbf{Responsable}\\
\midrule
09:00 & Ceremonia de inauguración y bienvenida & Dra. Daniela Robles\\
09:20 & Conferencia de apertura & TBD\\
\cellcolor{gray}{\textcolor{white}{\textbf{10:20}}} & \cellcolor{gray}{\textcolor{white}{\textbf{Regulatory Genomics and Bionformatics Lab}}} & \cellcolor{gray}{\textcolor{white}{\textbf{Dra. Alejandra Medina}}}\\
10:20 & Clinical, psychosocial, and demographic factors affect decisions in people with SLE & Domingo Martínez\\
10:35 & Single cell approaches for large-scale multiomic analysis in the framework of JAGUAR Project. & Diego Ramirez\\
\addlinespace
\cellcolor{lightgray}{10:50} & \cellcolor{lightgray}{Sesión de Posters 1 \& Coffee Break} & \cellcolor{lightgray}{NA}\\
\cellcolor{gray}{\textcolor{white}{\textbf{12:20}}} & \cellcolor{gray}{\textcolor{white}{\textbf{Mendelian Genomics and Precision Health Lab}}} & \cellcolor{gray}{\textcolor{white}{\textbf{Dra. Claudia Gonzaga}}}\\
12:20 & Allelic spectrum and prevalence of G6PD Deficiency in the Mexican Population & Aldair Henández\\
12:30 & Molecular and Clinical Variation Spectrum of Weiss-Kruszka Syndrome & Tania Sepulveda\\
\cellcolor{lightgray}{12:40} & \cellcolor{lightgray}{Modeling Dosage-Sensitive Genes Within a 9p13.3 Microduplication Associated with a Rare Neurodevelopmental Genomic Disorder} & \cellcolor{lightgray}{José Luis Tellez}\\
\addlinespace
\cellcolor{gray}{\textcolor{white}{\textbf{12:50}}} & \cellcolor{gray}{\textcolor{white}{\textbf{Lunch}}} & \cellcolor{gray}{\textcolor{white}{\textbf{NA}}}\\
14:10 & Evolutionary Systems Biology Lab & Dra. Mariana Gómez\\
14:10 & Bistability and Transcriptional Bursting: A Comparative Study of Noise-Driven Switching Dynamics & Diego Morales\\
\cellcolor{gray}{\textcolor{white}{\textbf{14:25}}} & \cellcolor{gray}{\textcolor{white}{\textbf{Mathematical modeling of negative feedback in maintenance of telomere homeostasis}}} & \cellcolor{gray}{\textcolor{white}{\textbf{Victoria Lelis}}}\\
14:40 & Cancer Genetics and Bioinformatics Lab & Dra. Daniela Robles\\
\addlinespace
14:40 & Metabolic and Lipid Alterations Induced by High-Fat Diets in MASLD Development & Claudia Gutiérrez\\
15:10 & Paleogenomics and Evolutionary Biology Lab & Dra. Federico Sánchez\\
\cellcolor{gray}{\textcolor{white}{\textbf{15:10}}} & \cellcolor{gray}{\textcolor{white}{\textbf{Camellos antiguos de la Cuenca de México incrementa el entendimiento de la historia evolutiva de los Camellos de América}}} & \cellcolor{gray}{\textcolor{white}{\textbf{Eduardo Arrieta}}}\\
15:25 & Explorando el microbioma de Mammuthus columbi mediante paleometagenómica & Santiago Rosas\\
15:40 & Fín del primer día & Dra. Daniela Robles\\
\bottomrule
\end{tabular}
\end{table}

\newpage

Viernes 6 de febrero, 2026

\begin{table}[!h]
\centering
\begin{tabular}{l>{\raggedright\arraybackslash}p{8cm}l}
\toprule
\textbf{Horario} & \textbf{Descripción} & \textbf{Responsable}\\
\midrule
\cellcolor{gray}{\textcolor{white}{\textbf{09:00}}} & \cellcolor{gray}{\textcolor{white}{\textbf{Population and Evolutionary Genomics Lab}}} & \cellcolor{gray}{\textcolor{white}{\textbf{Dra. María Ávila}}}\\
09:00 & Ancient Bacterial Disease and Viral Exchange from Holocene Pathogen Genomes in Patagonia & Florencia Alvarez\\
09:15 & Ancient HLA Variation and Predicted Immune Responses During Epidemics in Colonial Mexico & Walter Nicolas\\
\cellcolor{gray}{\textcolor{white}{\textbf{09:15}}} & \cellcolor{gray}{\textcolor{white}{\textbf{Evolutionary Dynamics of Red Complex Periodontal Pathogens in Colonial Mexico Revealed by Ancient DNA}}} & \cellcolor{gray}{\textcolor{white}{\textbf{Itzy Pérez}}}\\
09:30 & Statistical Genomics and Population Health Lab & Dr. Christopher Van Hout\\
\addlinespace
09:30 & Characterizing metabolic health and risk of cancer mortality using Principal Component Analysis in the Mexico City Prospective Study & Jair Contreras\\
\cellcolor{gray}{\textcolor{white}{\textbf{09:45}}} & \cellcolor{gray}{\textcolor{white}{\textbf{Factores demográficos y de estilo de vida que influyen en el IMC en MCPS}}} & \cellcolor{gray}{\textcolor{white}{\textbf{Diego Morales}}}\\
10:00 & Genome Evolution Lab & Dra. Lucía Morales\\
10:00 & Analysis of hybrid vigor in temperature gradients through experimental evolution & Ricardo Echavarria\\
\cellcolor{lightgray}{10:15} & \cellcolor{lightgray}{Allele frequency dynamics in metagenomes from agave fermentation} & \cellcolor{lightgray}{Renata Sandoval}\\
\addlinespace
\cellcolor{gray}{\textcolor{white}{\textbf{10:30}}} & \cellcolor{gray}{\textcolor{white}{\textbf{Sesión de Posters 3 \& Coffee Break}}} & \cellcolor{gray}{\textcolor{white}{\textbf{y}}}\\
12:00 & Ecology and Evolution Lab & Dr. Sur Herrera\\
12:00 & Diet-related gut microbiome differences in Systemic Lupus Erythematosus patients and healthy controls. & Yetel Ramírez\\
\cellcolor{lightgray}{12:15} & \cellcolor{lightgray}{Identifying ecological interactions mediated by biosynthetic gene clusters in the ocean microbiome} & \cellcolor{lightgray}{Andrea Zermeño}\\
\cellcolor{gray}{\textcolor{white}{\textbf{12:30}}} & \cellcolor{gray}{\textcolor{white}{\textbf{Lunch}}} & \cellcolor{gray}{\textcolor{white}{\textbf{y}}}\\
\addlinespace
13:50 & Computational Population Genetics Lab & Dr. Diego Ortega\\
\cellcolor{lightgray}{13:50} & \cellcolor{lightgray}{The impact of directional selection on patterns of genetic and phenotypic variation on a continuous space} & \cellcolor{lightgray}{Valeria Cabrera}\\
\cellcolor{gray}{\textcolor{white}{\textbf{14:00}}} & \cellcolor{gray}{\textcolor{white}{\textbf{Reconstrucción Paleogenómica del Desierto Chihuahuense: Explorando la Historia Evolutiva a través del sedaDNA}}} & \cellcolor{gray}{\textcolor{white}{\textbf{Laura Figueroa}}}\\
14:10 & Analysis of a Possible Relaxation of Natural Selection in Human Populations & Alejandra Marmolejo\\
14:10 & Sesión Informativa: VieRnes de Bioinformática & Dra. Evelia Coss\\
\addlinespace
14:20 & Clausura & Dra. Daniela Robles\\
14:30 & Evento Social & Emiliano Ferro?\\
\bottomrule
\end{tabular}
\end{table}

\newpage

\subsection{Oral presentation abstract}\label{oral-presentation-abstract}

\begin{quote}
\textbf{Inteligencia Artificial en la Educación} \\ 
\textit{Presenter: }Fulanito 1\\ 
\textit{Authors: }L. Martínez, A. Torres\\ 
\textit{Affiliations: }x\\ 
\textit{Abstract: }\\ 
Este trabajo explora el uso de IA para personalizar el aprendizaje en entornos virtuales.
\end{quote}

\begin{quote}
\textbf{Sostenibilidad y Energías Renovables} \\ 
\textit{Presenter: }Fulanito 2\\ 
\textit{Authors: }C. Ramírez, M. Díaz\\ 
\textit{Affiliations: }y\\ 
\textit{Abstract: }\\ 
Se presentan soluciones innovadoras para la generación de energía limpia en zonas urbanas.
\end{quote}

\begin{quote}
\textbf{Neurociencia y Aprendizaje} \\ 
\textit{Presenter: }Fulanito 3\\ 
\textit{Authors: }P. Herrera, G. Salas\\ 
\textit{Affiliations: }z\\ 
\textit{Abstract: }\\ 
El estudio analiza la relación entre procesos neuronales y estrategias de enseñanza.
\end{quote}

\newpage

\subsection{Distribución de los posters}\label{distribuciuxf3n-de-los-posters}

\subsubsection{Sesión de Posters 1 \& Coffee Break}\label{sesiuxf3n-de-posters-1-coffee-break}

\begin{table}[!h]
\centering
\begin{tabular}{rlll}
\toprule
Number & Presenter & Title & Laboratory\\
\midrule
1 & Fulanito1 & Titulo1 & Lab1\\
2 & Fulanito2 & Titulo2 & Lab2\\
3 & Fulanito3 & Titulo3 & Lab3\\
4 & Fulanito4 & Titulo4 & Lab4\\
5 & Fulanito5 & Titulo5 & Lab5\\
\addlinespace
6 & Fulanito6 & Titulo6 & Lab6\\
7 & Fulanito7 & Titulo7 & Lab7\\
8 & Fulanito8 & Titulo8 & Lab8\\
9 & Fulanito9 & Titulo9 & Lab9\\
10 & Fulanito10 & Titulo10 & Lab10\\
\addlinespace
11 & Fulanito11 & Titulo11 & Lab11\\
12 & Fulanito12 & Titulo12 & Lab12\\
13 & Fulanito13 & Titulo13 & Lab13\\
14 & Fulanito14 & Titulo14 & Lab14\\
15 & Fulanito15 & Titulo15 & Lab15\\
\addlinespace
16 & Fulanito16 & Titulo16 & Lab16\\
17 & Fulanito17 & Titulo17 & Lab17\\
18 & Fulanito18 & Titulo18 & Lab18\\
19 & Fulanito19 & Titulo19 & Lab19\\
20 & Fulanito20 & Titulo20 & Lab20\\
\bottomrule
\end{tabular}
\end{table}

\newpage

\subsubsection{Sesión de Posters 2 \& Coffee Break}\label{sesiuxf3n-de-posters-2-coffee-break}

\begin{table}[!h]
\centering
\begin{tabular}{rlll}
\toprule
Number & Presenter & Title & Laboratory\\
\midrule
1 & Fulanito1 & Titulo1 & Lab1\\
2 & Fulanito2 & Titulo2 & Lab2\\
3 & Fulanito3 & Titulo3 & Lab3\\
4 & Fulanito4 & Titulo4 & Lab4\\
5 & Fulanito5 & Titulo5 & Lab5\\
\addlinespace
6 & Fulanito6 & Titulo6 & Lab6\\
7 & Fulanito7 & Titulo7 & Lab7\\
8 & Fulanito8 & Titulo8 & Lab8\\
9 & Fulanito9 & Titulo9 & Lab9\\
10 & Fulanito10 & Titulo10 & Lab10\\
\addlinespace
11 & Fulanito11 & Titulo11 & Lab11\\
12 & Fulanito12 & Titulo12 & Lab12\\
13 & Fulanito13 & Titulo13 & Lab13\\
14 & Fulanito14 & Titulo14 & Lab14\\
15 & Fulanito15 & Titulo15 & Lab15\\
\addlinespace
16 & Fulanito16 & Titulo16 & Lab16\\
17 & Fulanito17 & Titulo17 & Lab17\\
18 & Fulanito18 & Titulo18 & Lab18\\
19 & Fulanito19 & Titulo19 & Lab19\\
20 & Fulanito20 & Titulo20 & Lab20\\
\bottomrule
\end{tabular}
\end{table}

\newpage

\subsubsection{Sesión de Posters 3 \& Coffee Break}\label{sesiuxf3n-de-posters-3-coffee-break}

\begin{table}[!h]
\centering
\begin{tabular}{rlll}
\toprule
Number & Presenter & Title & Laboratory\\
\midrule
1 & Fulanito1 & Titulo1 & Lab1\\
2 & Fulanito2 & Titulo2 & Lab2\\
3 & Fulanito3 & Titulo3 & Lab3\\
4 & Fulanito4 & Titulo4 & Lab4\\
5 & Fulanito5 & Titulo5 & Lab5\\
\addlinespace
6 & Fulanito6 & Titulo6 & Lab6\\
7 & Fulanito7 & Titulo7 & Lab7\\
8 & Fulanito8 & Titulo8 & Lab8\\
9 & Fulanito9 & Titulo9 & Lab9\\
10 & Fulanito10 & Titulo10 & Lab10\\
\addlinespace
11 & Fulanito11 & Titulo11 & Lab11\\
12 & Fulanito12 & Titulo12 & Lab12\\
13 & Fulanito13 & Titulo13 & Lab13\\
14 & Fulanito14 & Titulo14 & Lab14\\
15 & Fulanito15 & Titulo15 & Lab15\\
\addlinespace
16 & Fulanito16 & Titulo16 & Lab16\\
17 & Fulanito17 & Titulo17 & Lab17\\
18 & Fulanito18 & Titulo18 & Lab18\\
19 & Fulanito19 & Titulo19 & Lab19\\
20 & Fulanito20 & Titulo20 & Lab20\\
\bottomrule
\end{tabular}
\end{table}

\subsection{Posters abstracts}\label{posters-abstracts}

\begin{quote}
\textbf{Inteligencia Artificial en la Educación} \\ 
\textit{Presenter: }Fulanito 1\\ 
\textit{Authors: }L. Martínez, A. Torres\\ 
\textit{Affiliations: }x\\ 
\textit{Abstract: }\\ 
Este trabajo explora el uso de IA para personalizar el aprendizaje en entornos virtuales.
\end{quote}

\begin{quote}
\textbf{Sostenibilidad y Energías Renovables} \\ 
\textit{Presenter: }Fulanito 2\\ 
\textit{Authors: }C. Ramírez, M. Díaz\\ 
\textit{Affiliations: }y\\ 
\textit{Abstract: }\\ 
Se presentan soluciones innovadoras para la generación de energía limpia en zonas urbanas.
\end{quote}

\begin{quote}
\textbf{Neurociencia y Aprendizaje} \\ 
\textit{Presenter: }Fulanito 3\\ 
\textit{Authors: }P. Herrera, G. Salas\\ 
\textit{Affiliations: }z\\ 
\textit{Abstract: }\\ 
El estudio analiza la relación entre procesos neuronales y estrategias de enseñanza.
\end{quote}

\newpage

\begin{flushbottom}
\vspace*{\fill}
\begin{center}
\includegraphics[width=0.25\textwidth]{logo.png} % Ajusta el nombre y tamaño del archivo del logo
\hspace{1cm}
\parbox{0.7\textwidth}{
\textit{El comité organizador les desea que disfruten de los Días Académicos 2026, gracias por su confianza y apoyo.}
}
\end{center}
\vspace*{\fill}
\end{flushbottom}

\end{document}
